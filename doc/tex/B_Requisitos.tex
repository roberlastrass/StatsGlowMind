\apendice{Especificación de Requisitos}

\section{Introducción}

En este apéndice, se mostrarán y desarrollarán los objetivos y requisitos que debe cumplir la aplicación según han sido establecidos al comienzo del proyecto y los que se han ido añadiendo durante el proyecto.


\section{Objetivos generales}
Al comienzo del proyecto se establecieron los siguientes objetivos generales:
\begin{itemize}
\tightlist
    \item 
        Desarrollar una la aplicación web utilizando Angular como framework y Firebase como plataforma de alojamiento para la base de datos, autenticación de usuarios y servicio de hosting.
    \item
        Implementar la integración con una API con información de la NBA para recopilar datos en tiempo real. 
    \item 
        Gestionar la base de datos NoSQL (Firestore) para almacenar y gestionar los datos de la aplicación web.
    \item 
        Implementar un sistema de autenticación de usuarios.
    \item 
        Proporcionar información y estadísticas detalladas sobre la actual temporada de la NBA. 
    \item 
        Realizar análisis de rendimiento de jugadores mediante estadísticas y gráficas de datos.
    \item 
        Generar predicciones de resultados de partidos a partir de estadísticas y resultados de cada partido.
\end{itemize}


\section{Catálogo de requisitos}
En esta sección se exponen los requisitos funcionales y no funcionales del proyecto.

\subsection{Requisitos funcionales}
\begin{itemize}
\tightlist
    \item \textbf{RF1 - Login de usuarios:} La aplicación debe ser capaz de permitir a un usuario autenticarse y poder acceder a ciertas funcionalidades de la aplicación.
    \item \textbf{RF2 - Registro de usuarios:} La aplicación debe ser capaz de permitir a un usuario registrarse y almacenar sus datos en la base de datos.
    \item \textbf{RF3 - Logout de usuarios:} La aplicación debe permitir cerrar sesión a un usuario.
    \item \textbf{RF4 - Gestión de usuarios:} Un usuario con rol de administrador, debe poder gestionar el resto de usuarios, así como eliminar y modificar sus cuentas.
    \item \textbf{RF5 - Menú de la aplicación:} La aplicación debe contener un menú en el que puedas navegar por las distintas funcionalidades que tiene la aplicación.
    \item \textbf{RF6 - Main de la aplicación:} La aplicación debe de tener una página principal (main) en la que se muestren las funcionalidades de la aplicación de manera visual e intuitiva.
    \item \textbf{RF7 - Footer de la aplicación:} La aplicación debe contener un footer en el que se muestre más información sobre la aplicación.
        \begin{itemize}
        \tightlist
            \item \textbf{RF7.1 - Desarrollo de la aplicación:} En el footer debe contener información sobre herramientas y técnicas que se han utilizado para desarrollar la aplicación.
            \item \textbf{RF7.2 - Diseño de la aplicación:} En el footer debe contener información sobre herramientas y técnicas que se han utilizado para diseñar la aplicación.
            \item \textbf{RF7.3 - Acerca de la aplicación:} En el footer debe contener información sobre la aplicación.
            \item \textbf{RF7.4 - Redes sociales:} En el footer debe contener información sobre las redes sociales de la aplicación.
        \end{itemize}
    
    \item \textbf{RF8 - Mostrar información en tiempo real:} La aplicación debe mostrar información y estadísticas detalladas sobre la actual temporada de la NBA, en tiempo real.
        \begin{itemize}
        \tightlist
            \item \textbf{RF8.1 - Mostrar clasificación:} La aplicación debe poder mostrar la clasificación de la NBA junto con sus estadísticas.
            \item \textbf{RF8.2 - Mostrar playoffs:} La aplicación debe poder mostrar la eliminatoria de los playoffs de la NBA.
            \item \textbf{RF8.3 - Mostrar partidos:} La aplicación debe poder mostrar los partidos que se van a disputar y los que se han disputado, con sus resultados.
            \item \textbf{RF8.4 - Mostrar líderes:} La aplicación debe poder mostrar una tabla con los líderes de la NBA junto con sus estadísticas, y poder clasificar la lista según la categoría de líderes deseada.
        \end{itemize}
    \item \textbf{RF9 - Visualización de equipos:} La aplicación debe ser capaz de mostrar todos los equipos de todas las divisiones de cada conferencia de la NBA.
        \begin{itemize}
        \tightlist
            \item \textbf{RF9.1 - Estadísticas del equipo:} La aplicación debe poder mostrar mediante tablas, todas las estadísticas disponibles del equipo seleccionado.
            \item \textbf{RF9.2 - Jugadores del equipo:} La aplicación debe poder mostrar mediante una tabla, todos los jugadores junto con su información, del equipo seleccionado.
        \end{itemize}
    \item \textbf{RF10 - Análisis de jugadores:}
        \begin{itemize}
        \tightlist
            \item \textbf{RF10.1 - Mostrar jugadores:} La aplicación debe poder mostrar mediante una tabla todos los jugadores de la NBA y mediante cards los jugadores más destacados de la temporada.
            \item \textbf{RF10.2 - Gráficas y estadísticas del jugador:} La aplicación debe poder mostrar mediante una tabla las estadísticas medias de la temporada y analizar estas estadísticas mediante gráficas.
        \end{itemize}
    \item \textbf{RF11 - Predicción de partidos:} La aplicación debe ser capaz de proporcionar una predicción válida de un partido entre dos equipos seleccionados por el usuario.
    \item \textbf{RF12 - Cambio de idioma:} La aplicación debe poder modificar la configuración del idioma, de español a inglés y viceversa, si el usuario lo desea.
\end{itemize}

\subsubsection{Requisitos no funcionales}
\begin{itemize}
\tightlist
    \item \textbf{RNF1 - Usabilidad:} La aplicación debe tener una interfaz visual e intuitiva
    \item \textbf{RNF2 - Responsividad:} La aplicación debe ser capaz de adaptarse a los distintos tamaños de pantalla.
    \item \textbf{RNF3 - Internacionalización:} La aplicación debe poder configurar el idioma establecido.
    \item \textbf{RNF4 - Escalabilidad:} La aplicación debe estar preparada para desarrollar nuevas funcionalidades sin aumentar la carga de trabajo.
    \item \textbf{RNF5 - Mantenibilidad:} La aplicación debe permitir realizar mantenimientos y actualizaciones de forma eficaz. 
    \item \textbf{RNF6 - Seguridad:} La aplicación debe de proteger la integridad y confidencialidad de los datos de los usuarios y ocultar las claves y credenciales de acceso.
    \item \textbf{RNF7 - Disponibilidad:} La aplicación debe estar disponible para los usuarios todo el tiempo posible.
    
\end{itemize}



\begin{table}[p]
\section{Especificación de requisitos}

\hfill

\subsection{Casos de uso}

\hfill

	\centering
	\begin{tabularx}{\linewidth}{ p{0.21\columnwidth} p{0.71\columnwidth}}
		\toprule
		\textbf{CU-1}    & \textbf{Login de usuarios}\\
		\toprule
		\textbf{Requisitos asociados} & RF1 \\
		\textbf{Descripción}          & Permite a un usuario iniciar sesión y poder acceder a todas las funcionalidades de la aplicación y en caso de ser un usuario con rol de Administrador, también tiene acceso a la gestión de usuarios. \\
		\textbf{Precondición}         & Tener cuenta de Google si el usuario desea iniciar sesión mediante Google.\\
		\textbf{Acciones}             &
		\begin{enumerate}
			\def\labelenumi{\arabic{enumi}.}
			\tightlist
			\item El usuario accede a la página "\textit{Iniciar Sesión}".
			\item Opción 1: El usuario rellena el formulario introduciendo su correo electrónico y contraseña.
			Pulsa el botón de "\textit{Enviar}".
			\item Opción 2: El usuario pulsa el botón "\textit{Login With Google}".
			Selecciona su cuenta de Google.
		\end{enumerate}\\
		\textbf{Postcondición}        & El registro del usuario deber estar almacenado en la base de datos.		\\
		\textbf{Excepciones}          & 
		\begin{enumerate}
			\def\labelenumi{\arabic{enumi}.}
			\tightlist
			\item El usuario no exista en la base de datos (\textit{mensaje}).
			\item La contraseña no sea válida (\textit{mensaje}).
			\item El correo electrónico no sea válido (\textit{mensaje}).		
            \end{enumerate}\\
		\textbf{Importancia}          & Alta \\
		\bottomrule
	\end{tabularx}
	\caption{CU-1 Login de usuarios.}
\end{table}

\begin{table}[p]
	\centering
	\begin{tabularx}{\linewidth}{ p{0.21\columnwidth} p{0.71\columnwidth} }
		\toprule
		\textbf{CU-2}    & \textbf{Registro de usuarios}\\
		\toprule
		\textbf{Requisitos asociados} & RF2 \\
		\textbf{Descripción}          & Permite a un usuario crearse una cuenta mediante Correo electrónico/contraseña o mediante la cuenta de Google. \\
		\textbf{Precondición}         & Tener cuenta de Google si el usuario desea registrarse mediante Google. \\
		\textbf{Acciones}             &
		\begin{enumerate}
			\def\labelenumi{\arabic{enumi}.}
			\tightlist
			\item El usuario accede a la página "\textit{Iniciar Sesión}".
			\item El usuario selecciona a la opción "\textit{Registrarse}".
			\item Opción 1: El usuario rellena el formulario introduciendo su nombre de usuario, correo electrónico y contraseña, repite la contraseña y selecciona el rol de usuario.
			Pulsa el botón de "\textit{Enviar}".
			\item Opción 2: El usuario pulsa el botón "\textit{Register With Google}".
			Selecciona su cuenta de Google.
		\end{enumerate}\\
		\textbf{Postcondición}        &  El registro del usuario tiene que estar autenticado en Firebase. \\
		\textbf{Excepciones}          & 
		\begin{enumerate}
			\def\labelenumi{\arabic{enumi}.}
			\tightlist
			\item El usuario ya exista en la base de datos (\textit{mensaje}).
                \item Las contraseñas no coincidan (\textit{mensaje}).
			\item La contraseña no sea válida (\textit{mensaje}).
			\item El correo electrónico no sea válido (\textit{mensaje}).
            \end{enumerate}\\
		\textbf{Importancia}          & Alta \\
		\bottomrule
	\end{tabularx}
	\caption{CU-2 Registro de usuarios.}
\end{table}

\begin{table}[p]
	\centering
	\begin{tabularx}{\linewidth}{ p{0.21\columnwidth} p{0.71\columnwidth} }
		\toprule
		\textbf{CU-3}    & \textbf{Logout de usuarios}\\
		\toprule
		\textbf{Requisitos asociados} & RF3 \\
		\textbf{Descripción}          & Permite a un usuario cerrar sesión de la cuenta que tenga activa. \\
		\textbf{Precondición}         & Estar con la sesión iniciada. \\
		\textbf{Acciones}             & El usuario pulsa el botón "\textit{Cerrar Sesión}". \\
		\textbf{Postcondición}        & -	\\
		\textbf{Excepciones}          & -	\\
		\textbf{Importancia}          & Baja \\
		\bottomrule
	\end{tabularx}
	\caption{CU-3 Logout de usuarios.}
\end{table}

\begin{table}[p]
	\centering
	\begin{tabularx}{\linewidth}{ p{0.21\columnwidth} p{0.71\columnwidth} }
		\toprule
		\textbf{CU-4}    & \textbf{Gestión de usuarios}\\
		\toprule
		\textbf{Requisitos asociados} & RF4 \\
		\textbf{Descripción}          & Permite a un usuario con rol de administrador, modificar datos de las cuentas de usuario y eliminar usuarios. \\
		\textbf{Precondición}         & El usuario debe estar registrado y tener permisos de administrador.\\
		\textbf{Acciones}             &
		\begin{enumerate}
			\def\labelenumi{\arabic{enumi}.}
			\tightlist
			\item El usuario administrador accede a la página pulsando su "\textit{Nombre de Usuario}".
			\item Si el usuario administrador desea modificar los datos de un usuario, selecciona a la opción "\textit{Modificar}", cambia los datos deseados y pulsa en "\textit{Confirmar}".
			\item Si el usuario administrador desea eliminar a un usuario, selecciona la opción "\textit{Eliminar}" (aparecerá una ventana emergente para confirmar la operación) y pulsa en "\textit{Confirmar}".
		\end{enumerate}\\
		\textbf{Postcondición}        &  - \\
		\textbf{Excepciones}          &  El usuario administrador, no puede modificar o eliminar la cuenta de otro usuario administrador. \\
		\textbf{Importancia}          & Alta \\
		\bottomrule
	\end{tabularx}
	\caption{CU-4 Gestión de usuarios.}
\end{table}

\begin{table}[p]
	\centering
	\begin{tabularx}{\linewidth}{ p{0.21\columnwidth} p{0.71\columnwidth} }
		\toprule
		\textbf{CU-5}    & \textbf{Menú de la aplicación}\\
		\toprule
		\textbf{Requisitos asociados} & RF5 \\
		\textbf{Descripción}          & Permite al usuario navegar por las distintas funcionalidades que tiene la aplicación. \\
		\textbf{Precondición}         & Para poder navegar por todas las funcionalidades es necesario visualizar la aplicación desde una pantalla mediana/grande. \\
		\textbf{Acciones}             &
		\begin{enumerate}
			\def\labelenumi{\arabic{enumi}.}
			\tightlist
			\item El usuario puede acceder al main pulsando "\textit{StatsGlowMind}".
			\item El usuario puede acceder a las funcionalidades que contine el apartado NBA pulsando "\textit{NBA}".
            \item El usuario puede acceder a las funcionalidades que contine el apartado Equipos pulsando "\textit{Equipos}".
            \item El usuario puede acceder a las funcionalidades que contine el apartado Análisis pulsando "\textit{Análisis}".
            \item El usuario puede acceder a las funcionalidades que contine el apartado Predicción "\textit{Predicción}".
            \item El usuario puede acceder al inicio de sesión pulsando "\textit{Iniciar Sesión}".
            \item El usuario puede cerrar sesión pulsando "\textit{Cerrar Sesión}".
		\end{enumerate}\\
		\textbf{Postcondición}        &  - \\
		\textbf{Excepciones}          &  Visualizar la aplicación en versión móvil, en este caso solo hay acceso al main, inicio de sesión y cerrar sesión. \\
		\textbf{Importancia}          & Media \\
		\bottomrule
	\end{tabularx}
	\caption{CU-5 Menú de la aplicación.}
\end{table}

\begin{table}[p]
	\centering
	\begin{tabularx}{\linewidth}{ p{0.21\columnwidth} p{0.71\columnwidth} }
		\toprule
		\textbf{CU-6}    & \textbf{Main de la aplicación}\\
		\toprule
		\textbf{Requisitos asociados} & RF6 \\
		\textbf{Descripción}          & Permite al usuario acceder a todas las funcionalidades que tiene la aplicación de manera visual e intuitivo. \\
		\textbf{Precondición}         & El usuario debe estar registrado para acceder a todas las funcionalidades. \\
		\textbf{Acciones}             &
		\begin{enumerate}
			\def\labelenumi{\arabic{enumi}.}
			\tightlist
			\item El usuario puede acceder a las funcionalidades del apartado NBA pulsando "\textit{NBA}".
            \item El usuario puede acceder a las funcionalidades del apartado Equipos pulsando "\textit{Equipos}".
            \item El usuario puede acceder a las funcionalidades del apartado Análisis pulsando "\textit{Análisis}".
            \item El usuario puede acceder a las funcionalidades del apartado Predicción "\textit{Predicción}".
		\end{enumerate}\\
		\textbf{Postcondición}        &  - \\
		\textbf{Excepciones}          &  Un usuario invitado no puede acceder a todas las funcionalidades. \\
		\textbf{Importancia}          & Media \\
		\bottomrule
	\end{tabularx}
	\caption{CU-6 Main de la aplicación.}
\end{table}

\begin{table}[p]
	\centering
	\begin{tabularx}{\linewidth}{ p{0.21\columnwidth} p{0.71\columnwidth} }
		\toprule
		\textbf{CU-7}    & \textbf{Footer de la aplicación}\\
		\toprule
		\textbf{Requisitos asociados} & RF7, RF7.1, RF7.2, RF7.3, RF7.4 \\
		\textbf{Descripción}          & Permite al usuario acceder a información acerca de la aplicación. \\
		\textbf{Precondición}         & - \\
		\textbf{Acciones}             &
		\begin{enumerate}
			\def\labelenumi{\arabic{enumi}.}
			\tightlist
			\item El usuario puede acceder a la información del desarrollo de la aplicación.
            \item El usuario puede acceder a la información del diseño de la aplicación.
            \item El usuario puede acceder a la información acerca de la aplicación.
            \item El usuario puede acceder a las redes sociales de la aplicación.
		\end{enumerate}\\
		\textbf{Postcondición}        &  - \\
		\textbf{Excepciones}          &  - \\
		\textbf{Importancia}          & Baja \\
		\bottomrule
	\end{tabularx}
	\caption{CU-7 Footer de la aplicación.}
\end{table}

\begin{table}[p]
	\centering
	\begin{tabularx}{\linewidth}{ p{0.21\columnwidth} p{0.71\columnwidth} }
		\toprule
		\textbf{CU-8}    & \textbf{Mostrar información en tiempo real}\\
		\toprule
		\textbf{Requisitos asociados} & RF8, RF8.1, RF8.2, RF8.3, RF8.4 \\
		\textbf{Descripción}          & Permite al usuario visualizar información y estadísticas detalladas sobre la actual temporada de la NBA, en tiempo real. \\
		\textbf{Precondición}         & Tener acceso a la API. \\
		\textbf{Acciones}             &
		\begin{enumerate}
			\def\labelenumi{\arabic{enumi}.}
			\tightlist
            \item El usuario accede a la página "\textit{NBA}".
			\item El usuario puede visualizar la información y estadísticas de la clasificación pulsando en "\textit{Clasificación}".
            \item El usuario puede visualizar la información de los playoffs pulsando en "\textit{Playoffs}".
            \item El usuario puede visualizar la información de los partidos de la temporada pulsando en "\textit{Partidos}".
            \item El usuario puede visualizar la información y estadísticas de los lideres de la liga pulsando en "\textit{Líderes}".
		\end{enumerate}\\
		\textbf{Postcondición}        &  - \\
		\textbf{Excepciones}          &  Se hayan consumido todas las peticiones a la API. \\
		\textbf{Importancia}          &  Alta \\
		\bottomrule
	\end{tabularx}
	\caption{CU-8 Mostrar información en tiempo real.}
\end{table}

\begin{table}[p]
	\centering
	\begin{tabularx}{\linewidth}{ p{0.21\columnwidth} p{0.71\columnwidth} }
		\toprule
		\textbf{CU-9}    & \textbf{Mostrar partidos}\\
		\toprule
		\textbf{Requisitos asociados} & RF8.3 \\
		\textbf{Descripción}          & Permite al usuario visualizar los partidos que se van a disputar y los que se han disputado, junto con sus resultados. \\
		\textbf{Precondición}         & Tener acceso a la API. \\
		\textbf{Acciones}             &
		\begin{enumerate}
			\def\labelenumi{\arabic{enumi}.}
			\tightlist
			\item El usuario accede a los partidos pulsando en "\textit{Partidos}".
                \item El usuario visualiza los partidos jugados o por jugar de la fecha actual.
                \item El usuario puede visualizar partidos jugados o por jugar de la fecha deseada, seleccionado la fecha en el calendario.
		\end{enumerate}\\
		\textbf{Postcondición}        &  - \\
		\textbf{Excepciones}          &  Se hayan consumido todas las peticiones a la API. \\
		\textbf{Importancia}          &  Alta \\
		\bottomrule
	\end{tabularx}
	\caption{CU-9 Mostrar partidos.}
\end{table}

\begin{table}[p]
	\centering
	\begin{tabularx}{\linewidth}{ p{0.21\columnwidth} p{0.71\columnwidth} }
		\toprule
		\textbf{CU-10}    & \textbf{Mostrar líderes}\\
		\toprule
		\textbf{Requisitos asociados} & RF8.4 \\
		\textbf{Descripción}          & Permite al usuario visualizar los líderes de la NBA junto con sus estadísticas. \\
		\textbf{Precondición}         & Tener acceso a la API. \\
		\textbf{Acciones}             &
		\begin{enumerate}
			\def\labelenumi{\arabic{enumi}.}
			\tightlist
			\item El usuario accede a los líderes pulsando en "\textit{Líderes}".
            \item El usuario visualiza los líderes en puntos por partido de la NBA.
            \item El usuario puede visualizar líderes de otra categoría seleccionando la categoría deseada.
		\end{enumerate}\\
		\textbf{Postcondición}        &  - \\
		\textbf{Excepciones}          &  Se hayan consumido todas las peticiones a la API. \\
		\textbf{Importancia}          &  Alta \\
		\bottomrule
	\end{tabularx}
	\caption{CU-10 Mostrar líderes.}
\end{table}

\begin{table}[p]
	\centering
	\begin{tabularx}{\linewidth}{ p{0.21\columnwidth} p{0.71\columnwidth} }
		\toprule
		\textbf{CU-11}    & \textbf{Visualización de equipos}\\
		\toprule
		\textbf{Requisitos asociados} & RF9 \\
		\textbf{Descripción}          & Permite al usuario visualizar todos los equipos de la NBA \\
		\textbf{Precondición}         & El usuario debe estar registrado. \\
		\textbf{Acciones}             & El usuario accede a la página "\textit{Equipos}". \\
		\textbf{Postcondición}        &  Tener acceso a la base de datos. \\
		\textbf{Excepciones}          &  No cargue la base de datos. \\
		\textbf{Importancia}          &  Alta \\
		\bottomrule
	\end{tabularx}
	\caption{CU-11 Visualización de equipos.}
\end{table}

\begin{table}[p]
	\centering
	\begin{tabularx}{\linewidth}{ p{0.21\columnwidth} p{0.71\columnwidth} }
		\toprule
		\textbf{CU-12}    & \textbf{Estadísticas de equipos}\\
		\toprule
		\textbf{Requisitos asociados} & RF9.1 \\
		\textbf{Descripción}          & Permite al usuario visualizar mediante tablas, todas las estadísticas disponibles del equipo seleccionado. \\
		\textbf{Precondición}         & El usuario debe estar registrado. \\
		\textbf{Acciones}             & El usuario selecciona "\textit{Estadísticas}" del equipo deseado. \\
		\textbf{Postcondición}        &  Tener acceso a la base de datos. \\
		\textbf{Excepciones}          &  No cargue la base de datos. \\
		\textbf{Importancia}          &  Alta \\
		\bottomrule
	\end{tabularx}
	\caption{CU-12 Estadísticas de equipos.}
\end{table}

\begin{table}[p]
	\centering
	\begin{tabularx}{\linewidth}{ p{0.21\columnwidth} p{0.71\columnwidth} }
		\toprule
		\textbf{CU-13}    & \textbf{Jugadores del equipo}\\
		\toprule
		\textbf{Requisitos asociados} & RF9.2 \\
		\textbf{Descripción}          & Permite al usuario visualizar mediante una tabla, información general sobre todos los jugadores del equipo seleccionado. \\
		\textbf{Precondición}         & El usuario debe estar registrado. \\
		\textbf{Acciones}             & El usuario selecciona "\textit{Jugadores}" del equipo deseado. \\
		\textbf{Postcondición}        &  Tener acceso a la base de datos. \\
		\textbf{Excepciones}          &  No cargue la base de datos. \\
		\textbf{Importancia}          &  Alta \\
		\bottomrule
	\end{tabularx}
	\caption{CU-13 Jugadores del equipo.}
\end{table}

\begin{table}[p]
	\centering
	\begin{tabularx}{\linewidth}{ p{0.21\columnwidth} p{0.71\columnwidth} }
		\toprule
		\textbf{CU-14}    & \textbf{Mostrar jugadores}\\
		\toprule
		\textbf{Requisitos asociados} & RF10, R10.1 \\
		\textbf{Descripción}          & Permite al usuario visualizar mediante una tabla, todos los jugadores de la NBA y mediante cards los jugadores más destacados de la temporada. \\
		\textbf{Precondición}         & El usuario debe estar registrado. \\
		\textbf{Acciones}             & El usuario accede a la página "\textit{Análisis}". \\
		\textbf{Postcondición}        &  Tener acceso a la base de datos. \\
		\textbf{Excepciones}          &  No cargue la base de datos. \\
		\textbf{Importancia}          &  Alta \\
		\bottomrule
	\end{tabularx}
	\caption{CU-14 Mostrar jugadores.}
\end{table}

\begin{table}[p]
	\centering
	\begin{tabularx}{\linewidth}{ p{0.21\columnwidth} p{0.71\columnwidth} }
		\toprule
		\textbf{CU-15}    & \textbf{Gráficos y estadísticas del jugador}\\
		\toprule
		\textbf{Requisitos asociados} & RF10, R10.2 \\
		\textbf{Descripción}          & Permite al usuario visualizar mediante una tabla, las estadísticas medias de la temporada actual y analizar mediante gráficas de varios tipos las estadísticas del jugador. \\
		\textbf{Precondición}         &  El usuario debe estar registrado. \\
		\textbf{Acciones}             &  El usuario selecciona el jugador deseado. \\
		\textbf{Postcondición}        &  
            \begin{enumerate}
			\def\labelenumi{\arabic{enumi}.}
			\tightlist
			\item Tener acceso a la base de datos.
                \item El jugador seleccionado debe haber jugado al menos un partido de la temporada.
		\end{enumerate}\\
		\textbf{Excepciones}          &  No cargue la base de datos. \\
		\textbf{Importancia}          &  Alta \\
		\bottomrule
	\end{tabularx}
	\caption{CU-15 Gráficos y estadísticas del jugador.}
\end{table}

\begin{table}[p]
	\centering
	\begin{tabularx}{\linewidth}{ p{0.21\columnwidth} p{0.71\columnwidth} }
		\toprule
		\textbf{CU-16}    & \textbf{Predicción de partidos}\\
		\toprule
		\textbf{Requisitos asociados} & RF11 \\
		\textbf{Descripción}          & Permite al usuario seleccionar dos equipos para enfrentarse en un futuro partido y recibir una predicción de dicho partido. \\
		\textbf{Precondición}         &  El usuario debe estar registrado.\\
		\textbf{Acciones}             &  
            \begin{enumerate}
			\def\labelenumi{\arabic{enumi}.}
			\tightlist
			\item El usuario pulsa en "\textit{Predicciones}".
                \item El usuario selecciona los equipos deseados para el partido.
                \item El usuario pulsa el botón "\textit{Predecir Partido}".
		\end{enumerate}\\
		\textbf{Postcondición}        &  Se deben seleccionar los dos equipos. \\
		\textbf{Excepciones}          &  No cargue la función de Cloud Functions. \\
		\textbf{Importancia}          &  Alta \\
		\bottomrule
	\end{tabularx}
	\caption{CU-16 Predicción de partidos.}
\end{table}

\begin{table}[p]
	\centering
	\begin{tabularx}{\linewidth}{ p{0.21\columnwidth} p{0.71\columnwidth} }
		\toprule
		\textbf{CU-17}    & \textbf{Cambio de idioma}\\
		\toprule
		\textbf{Requisitos asociados} & RF12 \\
		\textbf{Descripción}          & Permite al usuario cambiar el idioma a español o inglés.\\
		\textbf{Precondición}         &  - \\
		\textbf{Acciones}             &  El usuario selecciona el idioma deseado desde un selector (ubicado en el pie de página). \\
		\textbf{Postcondición}        &  - \\
		\textbf{Excepciones}          &  - \\
		\textbf{Importancia}          &  Alta \\
		\bottomrule
	\end{tabularx}
	\caption{CU-17 Cambio de idioma.}
\end{table}

\clearpage

\subsection{Diagrama de casos de uso}

\imagen{anexos/diagrams/diagrama_casos_uso}{Diagrama de Casos de Uso}{1}