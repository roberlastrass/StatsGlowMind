\capitulo{6}{Trabajos relacionados}

En este apartado, se realiza un análisis de distintas aplicaciones web relacionadas con la temática del proyecto, es decir, Asociación Nacional de Baloncesto (NBA), dichas aplicaciones están compuestas de funcionalidades y características similares a las de StatGlowMind.  

El objetivo de este análisis es recoger ideas, funcionalidades y características adicionales de otros proyectos similares, para ampliar el desarrollo del proyecto y mejorarlo. Por esto mismo, realizar este estudio fue uno de los primeros pasos que realicé al comenzar el proyecto. 

\hfill

\section{Hoops Stats}
Hoops Stats \cite{hoopsstats} es una aplicación web que principalmente muestra toda clase de estadísticas de cada uno de los equipos y jugadores de la NBA, a nivel histórico. Además de esto, también tiene las distintas funcionalidades: 
\begin{itemize}
\tightlist
    \item Muestra todos los resultados de los partidos y los partidos restantes de toda la temporada de la NBA.
    \item Informa de datos curiosos como las rachas y récords de cada equipo y en base a estadísticas calcula los jugadores y equipos más mejorados con respecto a la temporada pasada.
    \item Notifica tanto la clasificación de la temporada de cada conferencia y playoffs como información de cada equipo y sus respectivos jugadores que lo componen.
\end{itemize}
\hfill
Todas estas funcionalidades están compuestas con tipos de filtrado como son: año de la temporada, posición del jugador, parte de la cancha, rookies, últimos partidos, mes, \ldots , y mucho más

\imagen{memoria/trabajos_relacionados/hoopsstats}{Estadísticas de los jugadores.}{1}

\hfill

\section{Basketball Reference}
Esta aplicación, Basketball Reference \cite{basketballreference}, es mucho más completa que la anterior ya que tiene las mismas funcionalidades de estadísticas, información de equipos y jugadores, partidos, sistema de filtrado, datos curiosos (récords, rachas, \ldots), etc.  

Y además contiene muchas más funcionalidades que aportan una gran interacción con el usuario:
\begin{itemize}
\tightlist
    \item Informa sobre noticias y tendencias de la actualidad.
    \item Realiza un análisis de datos exhaustivo de los jugadores a base de estadísticas. 
    \item Aparte de informar sobre la NBA, también muestra información de otras ligas de baloncesto como son ABA, WNBA, competiciones europeas y ligas internacionales. 
    \item Utilizan las redes sociales, como por ejemplo YouTube, para realizar vídeos de todo tipo (podcast, noticias, resúmenes de partidos, entrevistas, highlight, \ldots). 
    \item Contiene un juego interactivo para los usuarios, que consiste en adivinar jugadores que cumplan ciertas características. 
    \item Por último, también tienen un blog en el que tratan de varios deportes y categorías y puedes subscribirte para comentarios, comunicarte con otros usuarios y consultar dudas.
\end{itemize}

En general esta aplicación es muy completa ya que contiene información y estadísticas de todo tipo para que la puedan utilizar los analistas y entrenadores, y además proporciona muchas funcionalidades útiles e interesantes para los fanáticos del baloncesto. 

\imagen{memoria/trabajos_relacionados/basketballreference}{Juego interactivo.}{.5}

\hfill

\section{PROBALLERS}
PROBALLERS \cite{proballers} es una mezcla de las anteriores aplicaciones, es decir, muestra estadísticas e información de manera detallada como en Hoop Stats; y muestra más tipos de datos, noticias, distintas ligas de baloncesto, e interactúa más con el usuario como lo hace Basketball Reference. 

Sin embargo, esta aplicación web se nota mucho más profesional que las anteriores por el diseño y estilo que presenta y por las múltiples funcionalidades que contiene, como son: 
\begin{itemize}
\tightlist
    \item Muestra una galería de fotos de cada jugador de la liga. 
    \item Permite instalar widgets en tu dispositivo gratuitamente, sobre partidos de tu equipo favorito. 
    \item Puedes realizar seguimientos de super estrellas de la NBA. 
    \item Disponen de una aplicación de baloncesto disponible en iOS y Android.
    \item Tienen un diseño muy intuitivo y cómodo que aporta mucha más profesionalidad a la aplicación web. 
\end{itemize}
\hfill
Estas mezclas de funcionalidades de las dos aplicaciones, junto con las funcionalidades que acabo de mencionar, convierte esta aplicación web en la mejor opción de todas las aplicaciones vistas en este estudio. 

\imagen{memoria/trabajos_relacionados/proballers}{Página inicio.}{1}

\clearpage

\section{Tabla comparativa}
Tabla comparativa de las funcionalidades de las distintas páginas web sobre NBA: 


\tablaSmall{Tabla comparativa de funcionalidades}
{| >{\centering\arraybackslash}m{4cm} | >{\centering\arraybackslash}m{2cm} >{\centering\arraybackslash}m{2cm} >{\centering\arraybackslash}m{2cm} >{\centering\arraybackslash}m{2cm} |}{tablaComparativa}
{ \textbf{Funcionalidades}  & Stats Glow Mind & Hoops stats & Basketball Reference & Proballers \\}{ 
    Noticias & & & X & X\\
    Partidos y resultados & X & X & X & X\\
    Clasificación y Playoffs & X & X & X & X\\
    Equipos y Jugadores & X & X & X & X\\
    Líderes & X & & &\\
    Récords & & X & X & X\\
    Análisis de datos & X & & X & X\\
    Gráficos de datos & X & &  & \\
    Predicción de resultados & X & & & X\\
    Estadísticas en tiempo real & X & X & X & X\\
    Estadísticas históricas & & X & X & X\\
    Estadísticas de otros deportes o ligas & & & X & X\\
    Juego interactivo & & & X &\\
    Diseño visual e intuitivo & X & & & X\\
    Gratuito & X & X & X &\\
    Chat y comunidad & & & & \\
    Blog y artículos & & & X & \\
    Anuncios & & X & & X\\
    Redes sociales & & & X & X\\
    Varios idiomas & X & & & X\\
    Widget & & & & X\\
    App para Android o iOS & & & & X\\
} 