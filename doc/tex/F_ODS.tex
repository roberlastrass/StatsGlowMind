\apendice{Anexo de sostenibilización curricular}

\section{Introducción}
Este anexo incluye una reflexión personal sobre los aspectos de sostenibilidad abordados en el desarrollo de este trabajo. El objetivo es mostrar cómo se han adquirido y aplicado competencias de sostenibilidad durante el Trabajo de Fin de Grado, conforme a las directrices de la CRUE. A continuación, se detallan las competencias adquiridas y su aplicación práctica en el proyecto.

\section{Competencias de sostenibilidad}

\subsection{Conciencia y comprensión de la sostenibilidad}
Durante el desarrollo del proyecto, he adquirido una mayor conciencia sobre la importancia de la sostenibilidad en el ámbito tecnológico. Entender que las decisiones técnicas pueden tener un impacto significativo en el medio ambiente y la sociedad es crucial. Por ejemplo, la elección de servicios de almacenamiento y procesamiento en la nube fue influenciada por la búsqueda de proveedores con políticas de energía renovable y eficiencia energética. Firebase, como plataforma de backend, proporciona una infraestructura eficiente y escalable, minimizando el desperdicio de recursos.

\subsection{Diseño sostenible}
El diseño de la aplicación web de la NBA se ha realizado teniendo en cuenta principios de sostenibilidad. Se ha optimizado el código para reducir el consumo de recursos y mejorar la eficiencia. La implementación de algoritmos de Machine Learning, como el modelo Random Forest, se ha llevado a cabo de manera que el procesamiento se realice de forma eficiente, reduciendo la carga computacional y, por ende, el consumo energético.

\subsection{Uso responsable de los recursos}
La aplicación utiliza APIs externas para obtener datos en tiempo real. Para minimizar el número de peticiones y, por lo tanto, el consumo de recursos, se han implementado mecanismos de caché y almacenamiento temporal de datos. Esto no solo mejora la eficiencia de la aplicación, sino que también reduce la carga en los servidores externos, contribuyendo a un uso más responsable y sostenible de los recursos disponibles.

\subsection{Impacto social y accesibilidad}
La sostenibilidad también implica considerar el impacto social de la tecnología. En este proyecto, se ha trabajado para asegurar que la aplicación sea accesible para un público diverso. La implementación de funcionalidades de internacionalización (i18n) y la inclusión de traducciones con ngx-translate garantizan que usuarios de diferentes idiomas y regiones puedan utilizar la aplicación sin barreras lingüísticas. Además, se ha prestado atención a la accesibilidad web, asegurando que la aplicación cumpla con los estándares de accesibilidad y sea usable para personas con discapacidades.

\subsection{Educación y sensibilización}
El desarrollo del proyecto ha sido una oportunidad para educar y sensibilizar sobre la sostenibilidad en el ámbito tecnológico. Compartir las experiencias y los desafíos encontrados durante el desarrollo del proyecto con otros estudiantes y profesionales ha contribuido a crear una mayor conciencia sobre la importancia de la sostenibilidad en la tecnología. La documentación y el código del proyecto se han hecho disponibles de manera abierta para fomentar la colaboración y el aprendizaje colectivo.

\subsection{Evaluación y mejora continua}
Finalmente, la sostenibilidad es un proceso continuo que requiere evaluación y mejora constante. Durante el desarrollo del proyecto, se ha establecido un ciclo de retroalimentación para evaluar el impacto de las decisiones tomadas y buscar continuamente formas de mejorar. Esto incluye la monitorización del rendimiento de la aplicación y la búsqueda de nuevas tecnologías y prácticas que puedan hacer el proyecto más sostenible.

\section{Conclusión}
La integración de principios de sostenibilidad en el desarrollo de este proyecto ha sido profundamente educativa, permitiéndome valorar la importancia de considerar el impacto ambiental y social de las decisiones técnicas. A través de la optimización del uso de recursos, la elección de proveedores con políticas de energía renovable y la atención a la accesibilidad e impacto social, he aprendido que es posible crear soluciones tecnológicas avanzadas y responsables. Este enfoque no solo beneficia al medio ambiente y a la sociedad, sino que también añade valor al proyecto, haciéndolo más robusto y alineado con las expectativas actuales de sostenibilidad. Las competencias adquiridas durante este trabajo no solo son valiosas para el éxito de este proyecto, sino que también representan una base sólida para futuros proyectos profesionales. La sostenibilidad en la tecnología no es solo una tendencia, sino una necesidad para construir un futuro mejor y más equitativo.