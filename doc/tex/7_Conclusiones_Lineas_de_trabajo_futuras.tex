\capitulo{7}{Conclusiones y Líneas de trabajo futuras}

\section{Conclusiones}

Este proyecto me ha resultado muy útil tanto personalmente como profesionalmente. A lo largo del desarrollo de esta aplicación web dedicada a la NBA, he adquirido una serie de habilidades y conocimientos, marcados en los objetivos, que me han permitido crecer y madurar en diversas áreas.

Desde un punto de vista personal, trabajar en un proyecto que une mi pasión por la NBA con el desarrollo web ha sido muy gratificante. He podido profundizar en el análisis de datos deportivos, lo que me ha permitido obtener una comprensión más profunda de cómo se pueden utilizar las estadísticas para prever resultados y mejorar la experiencia de los aficionados al baloncesto.

En cuanto a las conclusiones técnicas, he aprendido a utilizar varias tecnologías que no se han visto en la carrera, como han sido Angular, TypeScript y Firebase. La experiencia con Angular me ha permitido dominar un framework de frontend muy potente, mientras que el uso de Firebase me ha proporcionado un conocimiento práctico sobre cómo gestionar bases de datos NoSQL, autenticar usuarios y desplegar funciones en la nube. La integración de Bootstrap y Angular Material me han permitido mejorar mis habilidades de diseño, mientras que la integración de Chart.js ha mejorado mis habilidades en la visualización de datos, pudiendo crear gráficos interactivos y dinámicos.

En resumen, este proyecto ha sido una gran experiencia para mi desarrollo técnico y profesional, proporcionándome una serie de competencias que serán valiosas en mi carrera profesional.

\section{Líneas de trabajo futuras}

El proyecto, aunque esta completo y funcional, me hubiera gustado añadir más funcionalidades y mejoras en el desarrollo de la aplicación. A continuación, se presentan algunas líneas de trabajo futuras que podrían llevarse a cabo para continuar desarrollando y mejorando la aplicación.

\subsection{Mejora del Análisis de Rendimiento con Algoritmos de Análisis de datos}
El análisis de rendimiento de los jugadores actualmente tan solo muestra las estadísticas a lo largo de la temporada mediante gráficos, esto permite ver la evolución del jugador, pero esto se podría mejorar implementando algoritmos de análisis de datos y Machine Learning, como redes neuronales profundas o métodos de clustering para segmentar y analizar datos con mayor precisión.

\subsection{Añadir más interacción con el usuario}
Para aumentar la interactividad de la aplicación, se podría ampliar el perfil del usuario para incluir más detalles personales, como jugadores y equipos favoritos y así implementar un sistema de notificaciones que informe a los usuarios sobre cuando juega su equipo, seguimiento de sus jugadores favoritos, ..., además de actualizaciones importantes, como resultados de partidos en tiempo real, predicciones y noticias relevantes.

\subsection{Creación de una Liga Fantasy de la NBA}
Una línea futura muy atractiva, la cual tenía mucho interés en desarrollarla, sería la creación de una Liga Fantasy de la NBA integrada dentro de la aplicación. Esto permitiría a los usuarios crear y gestionar sus propios equipos, competir con otros usuarios y recibir puntuaciones basadas en el rendimiento real de los jugadores. La integración de esta funcionalidad aumentaría bastante la experiencia del usuario.

\subsection{Agregar las estadísticas y récords históricos de cada temporada de la NBA}
Incorporar estadísticas y récords históricos de la NBA permitiría a los usuarios acceder a una base de datos completa y detallada de la NBA. Esta funcionalidad podría incluir gráficos y comparativas de estadísticas históricas, permitiendo a los usuarios explorar cómo han cambiado los patrones de juego a lo largo del tiempo.

\subsection{Mejorar la Seguridad de la Aplicación}
Aunque la seguridad y el manejo de errores ya están considerados en el proyecto actual, siempre se puede mejorar. Se podría implementar un sistema más robusto para la gestión de errores y excepciones, asegurando que la aplicación pueda manejar fallos inesperados sin comprometer la experiencia del usuario.
