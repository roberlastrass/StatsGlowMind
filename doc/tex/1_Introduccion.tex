\capitulo{1}{Introducción}

En el panorama deportivo contemporáneo, la tecnología desempeña un papel fundamental al proporcionar herramientas que no solo aumenta la experiencia del aficionado, sino que también aportan estadísticas e información valiosa para equipos, entrenadores y analistas.

En este contexto, este trabajo se centra en el desarrollo de una aplicación web dedicada a la Asociación Nacional de Baloncesto (NBA), una plataforma que va más allá de la simple exhibición de estadísticas y resultados, para ofrecer una serie de funcionalidades diseñadas para informar, analizar y predecir. 

\vfill

Históricamente, la comprensión del desempeño deportivo se basaba en datos estáticos y en análisis subjetivos. Sin embargo, con la incorporación de las tecnologías web y la disponibilidad de datos en tiempo real, surge la oportunidad de transformar radicalmente la forma en que se interpreta y se interactúa con el mundo del baloncesto profesional. 

Este proyecto aborda una serie de desafíos y oportunidades clave en el ámbito del análisis deportivo y la experiencia del usuario. Utilizando Angular y Firebase como marco tecnológico, junto con la integración de varias APIs, la aplicación ofrece al usuario de una forma dinámica y accesible, adentrarse al mundo de la liga de baloncesto más prestigiosa del mundo, la NBA. 

\vfill

Uno de los aspectos más destacados de esta aplicación es su capacidad para proporcionar análisis de jugadores en tiempo real, aprovechando las estadísticas de los jugadores en cada partido de la temporada, se realizan gráficas de análisis que muestran el rendimiento de los jugadores durante la temporada de la liga.

Además, este proyecto va más allá, al ofrecer predicciones sobre los resultados de los partidos futuros que puede seleccionar el usuario. Integrando un modelo con el algoritmo Random Forest de Machine Learning entrenado con datos históricos de la temporada actual, la aplicación proporciona a los usuarios la probabilidad de victoria de cada equipo, agregando más interés y emoción al seguimiento de la temporada de la NBA.

\hfill

En resumen, esta aplicación representa una mezcla entre el fanatismo por el baloncesto con la tecnología moderna. Al proporcionar una plataforma integral para la información, el análisis y la predicción, busca mejorar la experiencia de los aficionados y apoyar la toma de decisiones estratégicas a los equipos, entrenadores y analistas.

\vfill

\section{Estructura de la memoria}
\begin{itemize}
\tightlist
    \item
        \textbf{Introducción:} Descripción general del propósito y contexto del proyecto y estructura de la memoria y de los anexos.
    \item 
        \textbf{Objetivos del proyecto:} Listado de los objetivos generales y técnicos que se pretenden alcanzar con el proyecto, además de una lista de objetivos personales.
    \item 
        \textbf{Conceptos teóricos:} Explicación de los fundamentos teóricos importantes relacionados con el proyecto, para tener los conocimientos necesarios para su comprensión.
    \item 
        \textbf{Técnicas y herramientas:} Proporciona las técnicas y herramientas que han sido utilizadas en el desarrollo del proyecto. 
    \item 
        \textbf{Aspectos relevantes del desarrollo del proyecto:} Muestra los aspectos a destacar que han ocurrido en el desarrollo del proyecto.
    \item 
        \textbf{Trabajos relacionados:} Investigación previa de trabajos relacionados con el proyecto, realizando una comparativa de funcionalidades de cada aplicación.
    \item 
        \textbf{Conclusiones y líneas de trabajo futuras:} Conclusiones obtenidas al finalizar el proyecto y propuesta de posibles ideas de mejora para futuras actualizaciones.
\end{itemize}

\clearpage

\section{Estructura de los anexos}
\begin{itemize}
\tightlist
    \item 
        \textbf{Plan de proyecto software:} Planificación del proyecto, incluyendo por una parte la planificación temporal y por otra el estudio de viabilidad económica y legal.
    \item 
        \textbf{Especificación de requisitos:} Especificación de los requisitos según los objetivos establecidos al comienzo del proyecto y los que se han ido añadiendo durante el proyecto.
    \item 
        \textbf{Especificación de diseño:} Especificación de las distintas estructuras de diseño que se han empleado en el proyecto.
    \item 
        \textbf{Documentación técnica de programación:} Documentación de los conceptos más técnicos, como la estructura del proyecto, manual del programador, instalación de librerías, ejecución de la aplicación y la realización de las pruebas.
    \item 
        \textbf{Documentación de usuario:} Guía o manual de usuario para entender y utilizar la aplicación, sin problemas.
    \item 
        \textbf{Anexo de sostenibilización curricular:} Reflexión sobre los aspectos de la sostenibilidad que se abordan en el trabajo.
\end{itemize}