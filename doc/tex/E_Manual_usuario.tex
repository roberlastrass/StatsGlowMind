\apendice{Documentación de usuario}

\section{Introducción}

En este apéndice se explican los aspectos relacionados con la aplicación, por parte del usuario. Se muestran los requisitos de los usuarios y el manual de uso de la aplicación para el usuario.

\section{Requisitos de usuarios}

La aplicación web \textbf{StatsGlowMind} está desplegada para todo el mundo, lo que hace que no se necesiten muchos requisitos, tan solo es necesario contar con un dispositivo compatible como un ordenador, tablet o móvil. Y que dicho dispositivo cuente con un navegador web que contenga los siguientes requisitos mínimos recomendados de versiones:

\begin{itemize}
\tightlist
    \item Google Chrome: versión 87 o superior.
    \item Mozilla Firefox: versión 84 o superior.
    \item Microsoft Edge: versión 88 o superior.
    \item Safari: versión 14 o superior.
    \item Opera: versión 72 o superior.
\end{itemize}

\section{Instalación}

Como esta aplicación está desplegada para que cualquier usuario la pueda utilizar, no necesitan instalar nada, tan solo deben acceder al enlace \url{https://statsglowmindtfg.web.app/} desde un navegador.

\section{Manual del usuario}

En este manual se va a proporcionar una descripción detallada del uso de la aplicación, con el objetivo de facilitar al usuario a utilizar y comprender el funcionamiento de la aplicación.

Para ello, lo primero es acceder a la aplicación desde un navegador con el enlace mencionado anteriormente \url{https://statsglowmindtfg.web.app/} y comienza el uso de la aplicación.

\subsection{Inicio}

Inicialmente se observa la siguiente pantalla, en ella nos encontramos con una barra de navegación, la página principal (main) de la aplicación y el pie de página (footer) de la aplicación.

\imagen{anexos/interfaces/main}{Página de inicio.}{0.52}

\subsection{Barra de navegación}

La barra de navegación cuenta con el logotipo de la aplicación en el que si lo pulsas puedes acceder a la página principal, un menú desplegable con el que puedes acceder a las funcionalidades principales de la aplicación, un botón en el cual se puede iniciar sesión y otro botón para cerrar la sesión iniciada.

\imagen{anexos/user_manual/barra_navegacion}{Barra de navegación.}{1}

Esta barra va a estar presente en todas las páginas de la aplicación, por lo que podemos acceder a las funcionalidades mencionadas anteriormente en cualquier momento, excepto si utilizamos la aplicación en una pantalla pequeña como puede ser la de un móvil, en la cual la barra de navegación se vería así y tan solo podríamos acceder a la página principal, al inicio de sesión y a cerrar sesión.

\imagen{anexos/user_manual/barra_navegacion_movil}{Barra de navegación móvil.}{0.8}

Los botones de \textit{Iniciar Sesión} y \textit{Cerrar Sesión}, no tienen confusión, ya que si pasas el cursor por encima del botón te muestra el nombre de la función que realiza.

\imagen{anexos/user_manual/tooltip}{Información botones.}{0.3}

\subsection{Página principal}

En la página principal nos encontramos con dos Tabs, que son unos contenedores que contienen varias pestañas. El primer Tab contiene las pestañas de cada funcionalidad principal como son \textbf{NBA}, \textbf{EQUIPOS}, \textbf{ANÁLISIS} y \textbf{PREDICCIONES}, desde estas pestañas se puede acceder a la funcionalidad indicada. El segundo Tab contiene las pestañas con las funcionalidades de la NBA: \textbf{Clasificación}, \textbf{Playoffs}, \textbf{Partidos} y \textbf{Líderes}.

\imagen{anexos/user_manual/tabs}{Tabs página principal.}{0.9}

Justo debajo de estos Cards, podemos observar varios Cards (tarjetas) con sus descripciones, en los que cada uno de ellos, te lleva a una funcionalidad principal distinta.

\imagen{anexos/user_manual/cards}{Cards página principal.}{1}

También contiene un botón en la parte de arriba a la derecha de la página, en el cual contiene el nombre del usuario que ha iniciado sesión o en caso de que no se haya iniciado sesión, se mostraría como "\textit{Usuario Invitado}". Si pulsas en este botón y eres un usuario con permisos de administrador, accedes a la gestión de usuarios.

\imagen{anexos/user_manual/nombre_usuario}{Nombre de usuario.}{0.2}

\hfill

\subsection{Pie de página}

El pie de página se muestra en todas las páginas de la aplicación y en él se puede observar información sobre el \textbf{DESARROLLO}, \textbf{DISEÑO}, \textbf{ACERCA DE} y \textbf{REDES} de la aplicación. Cada información de cada uno de estos apartados te lleva a una página distinta mediante un enlace.

\imagen{anexos/user_manual/footer}{Pie de página.}{1}

Abajo de todo a la derecha se encuentra el selector de idioma, en el que puedes cambiar el idioma de español a inglés y viceversa.

\imagen{anexos/user_manual/idioma}{Selector de idioma.}{0.3}

Una vez seleccionas el idioma deseado, se cambia todo el contenido de la aplicación al idioma seleccionado. Así se ve el pie de página cuando seleccionas \textit{Inglés} como idioma de la aplicación.

\imagen{anexos/user_manual/footer_ingles}{Pie de página en inglés.}{1}

\hfill

\subsection{Inicio de Sesión}

Para navegar al inicio de sesión, se accede desde la barra de navegación pulsando en el botón de \textit{Iniciar Sesión}, como ya se mencionó. Una vez hemos accedido, podemos ver un formulario en el que se tiene que rellenar el email y la contraseña para poder iniciar sesión, o si se desea acceder con la cuenta de Google (si la tienes) tan solo hay que pulsar en el botón "\textit{Inicia sesión con Google"} y seleccionas tu cuenta de Google.

\imagen{anexos/user_manual/login}{Inicio de sesión.}{0.7}

Si el inicio de sesión ha sido correcto, la aplicación te redirigirá a la página principal (main), para que puedas continuar con el uso de la aplicación.

Si, por el contrario, el inicio no ha sido válido porque no se ha introducido bien algún dato o el usuario no está registrado aún, te saldrá el siguiente mensaje de error:
\imagen{anexos/user_manual/error_login}{Error de inicio de sesión.}{0.3}

En caso de que el usuario no se haya registrado todavía, tiene la opción de registrarse pulsando en el enlace "\textit{Regístrate}", que se encuentra en la parte de abajo del formulario.
\imagen{anexos/user_manual/boton_register}{Enlace para registrarse.}{0.3}

\hfill

\subsection{Registro de usuarios}

En el registro de usuarios, nos encontramos con otro formulario en el que hay que rellenar más campos: el nombre de usuario, el email, la contraseña, la contraseña repetida y el rol del usuario en el que a no ser que tengas permisos por el administrados, solo se podrá seleccionar el rol "\textit{Usuario}". También es posible saltarse todo esto si se desea registrarse con la cuenta de Google, pulsando en "\textit{Registrarse con Google"}.

\imagen{anexos/user_manual/register}{Registro de usuarios.}{0.6}

Cuando el usuario se registra exitosamente, se mostrará por pantalla el siguiente mensaje y te redirige a la página principal.
\imagen{anexos/user_manual/exito_register}{Enlace para registrarse.}{0.3}

Sin embargo, si existe algún error al rellenar el formulario, porque falta algún dato o el email está mal escrito o la contraseña es demasiado corta (min 6 caracteres), saldrá el siguiente mensaje de error:
\imagen{anexos/user_manual/error_register}{Error de registro.}{0.3}

Y si por algún casual todos los campos del formulario cumplen con las condiciones de registro, pero las contraseñas son distintas, aparece este mensaje de error:
\imagen{anexos/user_manual/error_password}{Contraseñas distintas.}{0.3}


\subsection{Gestión de usuarios}

Esta funcionalidad solo la pueden realizar los usuarios con rol de administrador. Para acceder a esta página, como comentamos en el apartado de la \textbf{Página Principal}, existe un botón en la parte superior a la derecha en el que aparece el nombre del usuario, que pulsando dicho botón se accede a la página de gestión de usuarios. En esta página aparecen todos los usuarios con todos sus datos, incluido la uid.

\imagen{anexos/user_manual/admin}{Gestión de usuarios.}{1}

\subsubsection{Modificación de usuarios}

El administrador podrá modificar cualquier usuario clicando el botón "\textit{MODIFICAR}", de tal forma que se mostrará una pantalla emergente en la que puedes cambiar el nombre y el email del usuario. Y pulsando en "\textit{Confirmar}" se guardan los cambios en la base de datos. En este ejemplo que voy a mostrar se modifica el email del usuario "\textit{lastras}":

\imagen{anexos/user_manual/user_update}{Modificación de usuarios.}{0.7}

Resultado de la modificación:
\imagen{anexos/user_manual/updated_user}{Usuario modificado.}{0.4}

\subsubsection{Eliminación de usuarios}

El administrador también podrá eliminar usuarios clicando el botón "\textit{ELIMINAR}", seguidamente aparece una pantalla emergente para confirmar si de verdad se desea eliminar al usuario, en la que se tendrá que pulsar en "\textit{Confirmar}" para eliminar al usuario seleccionado. En este caso se va a eliminar al usuario "\textit{lastras}":

\imagen{anexos/user_manual/user_delete}{Eliminación de usuarios.}{0.5}

Resultado de la modificación:
\imagen{anexos/user_manual/deleted_user}{Usuario eliminado.}{0.9}


\subsection{NBA}

Esta página que muestra estadísticas a tiempo real de la NBA está compuesta por 4 páginas más: \textbf{Clasificación}, \textbf{Playoffs}, \textbf{Partidos} y \textbf{Líderes}. Para acceder a cada una de estas funcionalidades, se puede realizar desde la barra de navegación (en el menú desplegable) o desde la página principal (en el Tab de la NBA o directamente en el Card NBA).

Se puede acceder a las distintas páginas de la NBA, desde este menú:
\imagen{anexos/user_manual/stats_menu}{Menú NBA.}{1}

\subsubsection{Clasificación}

Por defecto, si accedes a \textit{NBA} desde el Card de la página principal, navegas directamente a esta página, donde se muestra la clasificación de las dos conferencias de la NBA.

\imagen{anexos/user_manual/standings}{Clasificación de la NBA.}{0.72}

El icono naranja de las tablas, es un icono de información que te lleva al glosario de la aplicación para poder entender los términos de baloncesto empleados.
\imagen{anexos/user_manual/glossary_icon}{Icono del glosario.}{0.2}


\subsubsection{Playoffs}

En esta página se observa la tabla eliminatoria de los playoffs de la NBA.

\imagen{anexos/user_manual/playoffs}{Playoffs de la NBA.}{1}


\subsubsection{Partidos}

Desde esta página se pueden observar los próximos partidos que se van a disputar, como los partidos ya finalizados con sus resultados.

\imagen{anexos/user_manual/games}{Partidos de la NBA.}{1}

Para visualizar partidos de otras fechas, hay un calendario en el que se puede seleccionar el día deseado y observar los partidos jugados ese día. Todos estos partidos se muestran en la zona horaria UTC.

\imagen{anexos/user_manual/calendar}{Calendario de partidos.}{0.6}

Si en la fecha seleccionada no se jugó ningún partido, entonces aparecerá el siguiente texto:
\imagen{anexos/user_manual/no_games}{No se encontraron partidos.}{0.7}

\hfill

\subsubsection{Líderes}

Esta página muestra mediante una tabla las estadísticas de los jugadores líderes de la NBA según la categoría: Puntos, Rebotes, Asistencias, Robos, Tapones o Eficiencia.

\imagen{anexos/user_manual/leaders}{Líderes de la NBA.}{1}

Para cambiar de categoría, existe un selector en el que puedes seleccionar la categoría deseada:
\imagen{anexos/user_manual/category}{Categorías.}{0.5}

Al final de la tabla nos encontramos un paginador en el que podemos seleccionar cuantas filas queremos que se muestren y donde podemos cambiar de página:
\imagen{anexos/user_manual/paginator}{Paginador de la tabla.}{0.6}

\clearpage

\subsection{Equipos}

En la página de \textit{Equipos} se visualizan todos los equipos de la NBA y se puede acceder a las estadísticas de cada equipo pulsando en "\textit{ESTADÍSTICAS"} y a los jugadores de cada equipo pulsando en \textit{"JUGADORES"} del equipo deseado.

\imagen{anexos/user_manual/teams}{Equipos de la NBA.}{1}

\subsubsection{Estadísticas del equipo}

Desde esta página vemos las estadísticas del equipo incluyendo las estadísticas totales, en casa, en visitante, en partidos ganados y en partidos perdidos. También es posible acceder a los jugadores del mismo equipo pulsando al botón "\textit{JUGADORES"} ubicado en la parte superior de la derecha.
\imagen{anexos/user_manual/team_stats}{Estadísticas del equipo.}{1}

\subsubsection{Jugadores del equipo}

En esta página vemos los datos generales de los jugadores del equipo. Es posible acceder a las estadísticas del mismo equipo pulsando al botón "\textit{ESTADÍSTICAS"} ubicado en la parte superior de la izquierda.
\imagen{anexos/user_manual/team_players}{Jugadores del equipo.}{0.9}

\subsection{Análisis}

Cuando navegamos a la página de Análisis de los jugadores de la NBA, lo primero que nos aparece son unas Cards con los jugadores más destacados de la temporada. Justo debajo encontramos una tabla en la que se encuentran todos los jugadores de la NBA ordenados alfabéticamente.

\imagen{anexos/user_manual/analysis}{Análisis de jugadores.}{0.7}

En la parte superior de la tabla se encuentra un buscador, en el que podemos buscar un jugador en específico, como por ejemplo \textit{LeBron James}:
\imagen{anexos/user_manual/search}{Buscador de jugadores.}{1}

Para acceder al análisis del jugador, tan solo hay que clicar el jugador deseado o en caso de las Cards, pulsar en "\textit{ANÁLISIS}".

\subsubsection{Análisis del jugador}

En el análisis del jugador, primero se observa una tabla con las estadísticas totales de la temporada.
\imagen{anexos/user_manual/player_stats}{Estadísticas del jugador.}{1}

Seguidamente tenemos varias gráficas de datos, con los que podemos analizar el rendimiento del jugador.

Primero tenemos una gráfica de líneas en la que muestra las estadísticas principales por partido, como son los puntos, asistencias, rebotes, robos y tapones. En este tipo de gráficas es posible ocultar líneas de estadísticas pulsando directamente en el nombre de la estadística, por ejemplo, en BLK (Tapones).
\imagen{anexos/user_manual/stats_chart}{Gráfica de estadísticas por partido.}{1}

La siguiente gráfica es otra gráfica de líneas en la que se analizan los rebotes por partido, incluyendo rebotes defensivos, ofensivos y totales.
\imagen{anexos/user_manual/rebounds_chart}{Gráfica de rebotes por partido.}{1}

Continuamos con unas gráficas circulares, en las que se analizan los tiros de campo del jugador. El primer gráfico circular muestra los porcentajes de los puntos de media según el tipo de tiro de campo: tiro de dos puntos, triples y tiros libres. El resto de gráficos muestran el porcentaje de tiros de campo anotados y fallados, según el tipo de tiro.
\imagen{anexos/user_manual/field_goals_chart}{Gráfica de tiros de campo.}{0.95}

Por último, se muestra una gráfica de barras que indica la diferencia de puntos cuando el jugador está en la pista.
\imagen{anexos/user_manual/plus_minus_chart}{Gráfica de +/-.}{1}


\subsection{Predicciones}

Desde esta página se realizan las predicciones, para ello es necesario seleccionar los dos equipos que se van a enfrentar y después pulsar en "\textit{Predecir Partido}".

\imagen{anexos/user_manual/predict}{Predicciones de partidos.}{0.9}

Tras varios segundos esperando a que se calcule la predicción del partido, se mostrará en la parte inferior una gráfica circular en la que se muestran las probabilidades de victoria de cada equipo y al lado el equipo ganador según la predicción.

\imagen{anexos/user_manual/winner}{Equipo ganador.}{0.9}

En caso de que no se hayan seleccionado los dos equipos que se van a enfrentar, saldrá el siguiente mensaje de error:
\imagen{anexos/user_manual/error_predict}{Error de predicciones.}{0.5}

\subsection{Acerca de StatsGlowMind}

Esta página proporciona información acerca de la aplicación, así como, \textit{¿Qué es StatsGlowMind?}, \textit{Desarrollo}, \textit{Diseño} y \textit{Redes Sociales} de la aplicación.
\imagen{anexos/user_manual/about}{Acerca de StatsGlowMind.}{0.9}

\subsection{Glosario}
En el \textit{Glosario} de la aplicación se visualiza una tabla con los términos empleados en estadísticas y gráficos de baloncesto, de esta forma, el usuario que no conozca algún término, podrá entender toda la información de la aplicación sin problema.
\imagen{anexos/user_manual/glossary}{Glosario de la aplicación.}{0.7}
