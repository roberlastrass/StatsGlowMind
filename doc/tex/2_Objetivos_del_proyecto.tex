\capitulo{2}{Objetivos del proyecto}

En este apartado se detallan los objetivos que se persiguen con la realización del proyecto, distinguiendo entre objetivos generales, relacionados con los requisitos del software a construir; objetivos técnicos, necesarios para llevar a cabo los anteriores y objeticos personales, relacionados con las metas que me gustaría alcanzar al finalizar el proyecto.

\section{Objetivos generales}
\begin{itemize}
\tightlist
    \item 
        \textbf{Desarrollo de la aplicación web con Angular:} Desarrollar la lógica de la aplicación web utilizando el framework Angular, aprovechando su estructura de componentes y servicios para construir una aplicación modular y escalable.
    \item
        \textbf{Obtener y gestionar datos de la NBA:} Implementar la integración de una API con información de la NBA (API-NBA) para recopilar datos en tiempo real sobre clasificación, equipos, jugadores, partidos y estadísticas relevantes. 
    \item 
        \textbf{Integración de usuarios:} Implementar un sistema completo de autenticación de usuarios que permita el registro, inicio de sesión y modificación de perfiles, proporcionando una experiencia de usuario fluida y segura.
    \item 
        \textbf{Proporcionar información detallada sobre la NBA:} Mostrar información de la clasificación, playoffs, partidos y líderes de la sesión 23-24 de la NBA. Facilitar estadísticas exhaustivas y perfiles detallados de equipos y jugadores de la temporada actual.
    \item 
        \textbf{Realizar análisis de rendimiento de jugadores:} Implementar técnicas de análisis de datos y la librería Chart.js, para evaluar mediante gráficas el desempeño y eficiencia de los jugadores en diferentes aspectos del juego, como puntos, asistencias, rebotes, entre otros.
    \item 
        \textbf{Generar predicciones de resultados de partidos:} Aplicar el método Random Forest de Machine Learning que estudie las estadísticas y resultados de cada partido jugado por cada equipo y entrene un modelo para predecir los resultados de futuros partidos y proporcionar a los usuarios las probabilidades de victoria sobre los posibles ganadores.
    \item 
        \textbf{Facilitar la interacción de los usuarios con la aplicación:} Desarrollar un diseño de interfaces visual e intuitivo en las funcionalidades a las que puedan acceder los usuarios para facilitar el uso de la aplicación.
\end{itemize}

\section{Objetivos técnicos}
\begin{itemize}
\tightlist
    \item
        \textbf{Integración con Firebase:} Utilizar Firebase como plataforma de alojamiento para la base de datos, almacenamiento de archivos, autenticación de usuarios, servicio de hosting y funciones en la nube, para la aplicación web.
    \item 
        \textbf{Gestión de base de datos NoSQL Firestore:} Adquirir conocimientos y habilidades en el uso de Firestore de Firebase como una base de datos NoSQL para almacenar y gestionar los datos de la aplicación web.
    \item 
        \textbf{Implementación de Bootstrap:} Utilizar Bootstrap para mejorar la estética y la usabilidad de la aplicación web, haciéndola así una aplicación resposive para poder acceder a ella desde cualquier dispositivo sin perder información.
    \item 
        \textbf{Implementación de Angular Material:} Integrar Angular Material en la aplicación web para aprovechar su biblioteca de componentes predefinidos y estilizados, garantizando así una interfaz de usuario moderna y consistente en toda la aplicación.
    \item 
        \textbf{Implementación de Firebase Cloud Functions:} Inicializar Firebase Cloud Functions en mi proyecto, para utilizarlo como backend del proyecto y poder realizar la función de predicción de partidos mediante Python.
    \item 
        \textbf{Seguridad de datos:} Aplicar medidas de seguridad para proteger la integridad y confidencialidad de los datos de los usuarios utilizando Firebase Autentication, y añadir las claves y credenciales de acceso en .gitignore para proteger la seguridad de la aplicación.
    \item 
        \textbf{Internacionalización del proyecto:} Realizar la internacionalización del proyecto, para que los usuarios tengan la posibilidad de cambiar de idioma siempre que lo deseen.
    \item 
        \textbf{Documentación del proyecto:} Elaborar una documentación completa que permita facilitar la comprensión y mantenimiento del proyecto.
\end{itemize}

\section{Objetivos personales}
\begin{itemize}
\tightlist
    \item
        \textbf{Profundizar en el desarrollo frontend con Angular:} Aumentar mi experiencia y habilidades en el desarrollo de aplicaciones web frontend utilizando Angular.
    \item 
        \textbf{Dominar el uso de Firebase como plataforma de desarrollo web:} Adquirir un conocimiento sólido sobre Firebase, tanto en su utilización para alojamiento de bases de datos, alojamiento de archivos y autenticación de usuarios, como en la implementación de funciones de hosting y cloud functions. 
    \item 
        \textbf{Aprender técnicas de Machine Learning:} Investigar y aplicar técnicas de Machine Learning en el contexto de predicciones de resultados de partidos en la NBA.
    \item 
        \textbf{Adquirir experiencia en la integración de APIs externas:} Aprender a integrar y trabajar con APIs externas, para obtener y procesar datos relevantes de manera eficaz y segura dentro de la aplicación web.
    \item 
        \textbf{Mejorar habilidades de diseño:} Incrementar y mejorar mis habilidades en diseño de interfaces de usuario, asegurando que la aplicación sea intuitiva, atractiva y fácil de usar para los usuarios finales.
    \item 
        \textbf{Desarrollar habilidades de resolución de problemas:} Reforzar mis habilidades en la identificación y resolución de problemas técnicos y desafíos durante el desarrollo del proyecto.
    \item 
        \textbf{Documentar adecuadamente el proceso de desarrollo:} Asegurar una documentación completa y detallada del proceso de desarrollo.
\end{itemize}